\section{Bevezető}

A tárgyhoz kapcsolatos feladatom egy bináris csillagrendszer vizsgálata volt. A vizsgálathoz szükséges szimulációt C++ nyelven írtam meg és Jupyter Notebook (Python 3) környezetben alkottam meg az ábrákat.

A szimuláció megfelelő megírása előtt szükséges némi fizikai háttér. Mivel ebben az esetben egy három test problémával állunk szemben, így ennek sok esetben nincs analitikus megoldása, tehát numerikus számítások segítségével kell tovább haladnunk. Ehhez szükséges Runge-Kutta 4 lépéses integrátort kellett alkalmaznom mely várja a megfelelő egyenletek megfogalmazását, így számunkra a fladat a következő: ebben az esetben a gravitációs erőtörvényt kell jól definiálni. A szimulációt érdemes tehát kis időléptékkel léptetni a pontosság miatt.

A gravitációs erőtörvényt a következőképpen fogalmazhatjuk meg $m_i$ és $m_j$ között:
\begin{equation}
    F_{ij} = \gamma \cdot \frac{m_i \cdot m_j}{(r_j - r_i)^2} \cdot \frac{r_j - r_i}{|r_j - r_i|}
\end{equation}
F$_{ij}$ osztása m$_j$-vel eredményezi az m$_j$-re ható m$_i$ erő gyorsítását. Az erők szuperpozició elve alapján pedig fel lehet összegezni erre. Kinetikus energiát pedig a következőképpen szánolhatunk.
\begin{equation}
    T_i = \frac{1}{2} \cdot m_i \cdot v_i^2
\end{equation}
Potenciális energiát pedig a következő formulával:
\begin{equation}
    U_i = \sum_{j=0; j\neq i}^N \gamma \cdot \frac{m_j \cdot m_i}{||r_j - r_i||}
\end{equation}
Ha ezkkel felösszegzünk, akkor a rendszer energia állapotát vizsgálhatjuk.