\section{Leírás}

A szimulációhoz szükséges tehát megfogalmaznunk, hogy milyen módon tároljuk az adatokat, milyen koordinátarendszert használunk és hogy milyen kinézetű legyen a kimenet. Legkényelmesebb választás a Descartes koordinátarendszer, melyhez a következő bemeneti formá választottam.

\begin{table}[ht!]
    \centering
    \begin{tabular}{|c|c|c|c|c|c|c|}  \hline 
        Tömeg & $r_1$ & $r_2$ & $r_3$ & $v_1$ & $v_2$ & $v_3$  \\ \hline 
        ... & ... & ... & ... & ... & ... & ... \\  
    \end{tabular}
    \caption{Bemeneti fájl várt struktúrája.}
    \label{tab:input}
\end{table}
Ezen adatokkal menthető is az alapállapot, mint t=0 időpontban mért rendszer. Viszont kérdésesek a mennyiségek. Mivel egy csillagrendszer szimulációját hajtom végre, így csillagászatban használt mennyiségek választása hasznos. Tehát a távolság mérése csillagászati egységekben (AU) történik, az idő skála pedig években mérendő. Ennek következtében minden szükséges állandót át kell alakítani, hogy ilyen dimenziójú mennyiségek alkossák. 
Továbbá érdemes ekkor eldönteni is, hogy milyen formában történjen a rendszer egyes állapotainak mentése. Számomra megfelelőnek tünt, hogy minden egyes égitest egy egyéni fáljba legyen kimentve. A programm fejlesztése során problémába ütközött a fájlok folytonos nyitvatartása, így minden állapotnak a mentésénél a program újra és újra megnyitja majd bezárja azokat. (mivel az ofstream egy olyan pointert ad, ami a fálj elejére mutat, így külön parancsban kell a fájl végére mutató pointert oda adni neki). A kimenő fájlok tehát a következő formátumúak.
\begin{table}[ht!]
    \centering
    \begin{tabular}{|c|c|c|c|c|c|c|c|c|}  \hline 
        Idő & $r_1$ & $r_2$ & $r_3$ & $v_1$ & $v_2$ & $v_3$ & $T$ & $U$  \\ \hline 
        ... & ... & ... & ... & ... & ... & ...& ... & ... \\  
    \end{tabular}
    \caption{Kimeneti fálj struktúrája. T és U a kinetikus és potenciális energiát jelölik.}
    \label{tab:output}
\end{table}
Így minden fáljban szerepelni fog háromszor az állapot időpontja. Ha ezen állapotokat egy fáljba rendezzük, akkor lehetőség van ezek összekeverésére, így értelmesebb a szétválasztás. A program a fájlokat az inicializáló fájl alapján, a sorok sorrendjével megfelelően kezeli, így az "output" mappában "body1.csv","body2.csv",...stb. fog megjelenni.
Ezek után már csak egy lépésben szükséges a régi is új állapotokat kell tárolni, ami egyszerűen kivitelezhető. Mivel három test van, így mindhárom testre a többi által ható erőt kell venni, de mindig a régebbi állapot által. Ha ez megtörtént, akkor a régi állapot felülbírálódik az újjal, majd mentve lesz.